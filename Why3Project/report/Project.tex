% Header
\documentclass[a4paper]{article}%      autres choix : book, report
\usepackage[utf8]{inputenc}%           gestion des accents (source)
\usepackage[T1]{fontenc}%              gestion des accents (PDF)
\usepackage[francais]{babel}%          gestion du français
\usepackage{textcomp}%                 caractères additionnels
\usepackage{mathtools,amssymb,amsthm}% packages de l'AMS + mathtools
\usepackage{lmodern}%                  police de caractère
\usepackage[top=2cm,left=2cm,right=2cm,bottom=2cm]{geometry}%     gestion des marges
\usepackage{graphicx}%                 gestion des images
\usepackage{array}%                    gestion améliorée des tableaux
\usepackage{calc}%                     syntaxe naturelle pour les calculs
\usepackage{titlesec}%                 pour les sections
\usepackage{titletoc}%                 pour la table des matières
\usepackage{fancyhdr}%                 pour les en-têtes
\usepackage{titling}%                  pour le titre
\usepackage{enumitem}%                 pour les listes numérotées
\usepackage{hyperref}%                 gestion des hyperliens
\usepackage{minted}
\usemintedstyle{vs}

\hypersetup{pdfstartview=XYZ}%         zoom par défaut

\setlength{\droptitle}{-5em}   % This is your set screw
\title{\vspace{1.5cm}Projet Why3}
\author{Mickaël LAURENT}
\date{\vspace{-5ex}}

\pagenumbering{gobble}
\DeclarePairedDelimiter\ceil{\lceil}{\rceil}
\DeclarePairedDelimiter\floor{\lfloor}{\rfloor}

\begin{document}

	\maketitle

	\section{Implémentation et preuves}

	\paragraph{0. axioms}
	Je n'ai eu besoin que de l'axiome supplémentaire proposé dans le sujet:
	\[\forall xy, 0 < x < y \implies \log(x) < \log(y)\]

	\paragraph{1. power2}
	J'avais initialement implémenté cette fonction de manière itérative (code toujours présent mais commenté),
	mais j'ai finalement opté pour une version récursive, à la fois plus concise à implémenter et à prouver.
	En effet, la bibliothèque standard définie elle-même power de manière récursive, rendant la preuve de mon implémentation immédiate.
	Il a seulement été nécessaire d'indiquer le variant: l'argument $l$ en entrée de la fonction.

	\paragraph{2. shift\_left}
	J'ai opté pour l'implémentation suivante:  \mintinline{ocaml}{z * (power2 l)}.\\
	Elle traduit littéralement la spécification de la fonction: \mintinline{ocaml}{result = z * (power 2 l)}.

	\paragraph{3. ediv\_mod}
	Mon implémentation est itérative et utilise la méthode d'euclide, comme celle faite en exercice durant l'année.
	En revanche, ici $x$ peut être négatif, ainsi j'utilise deux implémentations différentes selon le cas:
	si $x \geq 0$ alors je commence avec un reste $r$ positif auquel je soustrait $y$ à chaque itération, sinon, si $x < 0$,
	je commence avec un $r$ négatif auquel j'ajoute $y$ à chaque itération.\\
	La terminaison se prouve facilement en spécifiant le variant: $r$ dans un cas, $-r$ dans l'autre.\\
	La correction se prouve également sans difficulté en utilisant l'invariant adéquate, par exemple pour le cas $x < 0$:
	\mintinline{ocaml}{x = !q * y + !r /\ !r < y}.

	\paragraph{4. shift\_right}
	J'utilise l'implémentation suivante: \mintinline{ocaml}{let d,_ = ediv_mod z (power2 l) in d},
	qui traduit directement la spécification voulue: \mintinline{ocaml}{result = ED.div z (power 2 l)}.
	
	\paragraph{5. isqrt}
	Là encore j'utilise l'implémentation itérative vue en cours:
	\begin{minted}{ocaml}
let count = ref 0 in
let sum = ref 1 in
while !sum <= n do
	invariant { 0 <= !count /\ !count * !count <= n /\ !sum = (!count+1)*(!count+1) }
	variant { n - !sum }
	count := !count + 1;
	sum := !sum + 2 * !count + 1
done ; !count
	\end{minted}
	La terminaison est immédiate étant donné le variant,
	mais la correction (\mintinline{ocaml}{result = floor (sqrt (from_int n))}) nécessite tout de même
	de spécifier quelques trivialités après la boucle while (voir dans le code).\\

	NOTE: je n'ai pas eu besoin du lemme suggéré (euclid\_uniq) pour les 5 questions précédentes.

	\paragraph{6-12. some properties} La quasi-totalité des propriétés de l'énoncé ont été prouvées
	immédiatement par Z3, excepté \mintinline{ocaml}{_B_9} et \mintinline{ocaml}{_B_11} qui ont
	nécessité une étape intermédiaire (lemmes \mintinline{ocaml}{_B_9_aux} et \mintinline{ocaml}{_B_11_aux}).
	\begin{minted}{ocaml}
lemma _B_6 : forall n. _B n >. 0.
lemma _B_7 : forall n, m. (_B n) *. (_B m) = _B (n + m)
lemma _B_8 : forall n. _B n *. _B (-n) = 1.
lemma _B_9_aux : forall n. sqrt (_B (2*n)) = _B n
lemma _B_9 : forall a,n. 0. <=. a -> sqrt (a *. _B (2*n)) = (sqrt a) *. (_B n)
lemma _B_10 : forall y. 0 <= y -> _B y = from_int (power 4 y)
lemma _B_11_aux : forall y. (pow b y)*.(pow b (-.y)) = 1. (* Same as B_8 but for reals *)
lemma _B_11 : forall y. y < 0 -> _B y = inv (from_int (power 4 (-y)))
lemma _B_12 : forall y. 0 <= y -> power 2 (2 * y) = power 4 y
	\end{minted}

	\paragraph{13. framing} La définition de l'énoncé est meilleure que $|x - p(B^{-n})| < B^{-n}$ car elle
	est divisée en deux inégalités plus simples que l'on peut prouver séparemment.
	Ces deux inégalités auraient de toute façon dû être prouvées avant de pouvoir prouver $|x - p(B^{-n})| < B^{-n}$
	car une disjonction de cas est requise par la valeur absolue.

	\paragraph{14. addition} La preuve compliquée ici est \mintinline{ocaml}{compute_round}.
	Il faut montrer \mintinline{ocaml}{framing z result n}. J'ai choisi de montrer les deux inégalités
	correspondantes séparemment (les preuves sont similaires, mais je trouvais cela trop chargé de tout faire
	en une seule fois). Pour chaque inégalité, on part de l'encadrement de \mintinline{ocaml}{round_z_over_4 zp}
	(assuré par la spécification de denière fonction), on multiple par $B^{-n}$ des deux cotés et on simplifie
	jusqu'à obtenir l'encadrement voulu:
	\begin{minted}{ocaml}
assert { (from_int res) <=. ((from_int zp +. 2.) *. _B (-1)) } ;
assert { (from_int res) *. (_B (-n)) <=. ((from_int zp +. 2.) *. _B (-1)) *. (_B (-n)) } ;
...
assert { (from_int res) *. (_B (-n)) <=. lb +. (_B (-n)) } ;
assert { (from_int res -. 1.) *. (_B (-n)) <=. lb } ;
	\end{minted}

	\paragraph{15. compute\_neg} Aucune difficulté ici, on obtient un encadrement valide de $-x$ simplement
	en prenant la négation d'un l'encadrement de $x$ de même précision.

	\paragraph{16. compute\_sub} Aucune difficulté non plus, la soustraction de $x$ par $y$
	étant équivalente à l'addition de $x$ et de l'opposé de $y$. Il faut simplement faire attention à
	demander un encadrement de $x$ et $y$ de précision $n+1$, car l'addition fait perdre un niveau de précision.

	\paragraph{16,5. compute\_cst} La preuve de cette fonction n'est pas immédiate, bien que ne posant aucune difficulté mathématique.
	En particulier, les cas $n<0$ et $n>0$ nécessitent quelques étapes intermédiaires.
	J'ai encore une fois prouvé chaque inégalité de l'encadrement séparemment, chacune se simplifiant facilement en une inégalité triviale.

	\paragraph{17. boundaries for sqrt} Pour prouver l'encadrement de l'énoncé, j'ai commencé par prouver un lemme plus simple:
	$\sqrt{n+1}-\sqrt{n}\leq 1$. Il se prouve facilement en montrant que $(\sqrt{n+1}-\sqrt{n})^2\leq 1$.
	A partir de ce lemme, la preuve de $\floor*{\sqrt{n}} \leq \floor*{\sqrt{n-1}}+1$ est immédiate.\\
	En revanche, la preuve de $\ceil*{\sqrt{n+1}}-1 \leq \floor*{\sqrt{n}}$ est un peu plus astucieuse.
	J'ai séparé la preuve en deux cas, selon que $n+1$ soit un carré parfait ou non. Dans le premier cas,
	on peut conclure directement grâce au lemme et au fait que $\ceil*{\sqrt{n+1}}=\sqrt{n+1}$.
	Dans le second cas, on peut montrer par l'absurde que $\floor*{\sqrt{n+1}}\leq \floor*{\sqrt{n}}$
	(en effet, $\floor*{\sqrt{n+1}} > \floor*{\sqrt{n}} \implies n < (\floor*{\sqrt{n+1}})^2 < n+1$),
	puis conclure sans difficulté.

	\paragraph{18. compute\_sqrt} Bien qu'il m'ai fallu plusieurs étapes intermédiaires, cette preuve est relativement
	facile. Pour le case du \mintinline{ocaml}{else}, on part de l'encadrement initial de x, on passe à la racine et on se sert du résultat
	précédent pour obtenir l'encadrement voulu (on peut temporairement multiplier le tout par $B^n$ pour pouvoir appliquer plus
	facilement le résultat précédent).

	\paragraph{19. interp} Aucune difficulté pour cette fonction. Il faut simplement bien penser à la déclarer \mintinline{ocaml}{ghost},
	car les opérations sur les réels ne sont pas calculables.

	\paragraph{20. wf\_term} Le seul cas intéressant ici est le cas de la racine. Pour ce cas il faut vérifier que l'interpretation
	de l'argument est positif ET qu'il est un terme valide (le fait que le terme en argument ait une intrepretation positive
	n'implique pas qu'il soit valide).
	\begin{minted}{ocaml}
Sqrt t    -> (wf_term t) && (interp t >=. 0.)
	\end{minted}

	\paragraph{21. compute} L'implémentation de cette fonction est facile puisqu'il suffit d'appeler les fonctions
	définies précedemment pour chaque cas. Il faut simplment faire attention à cacluler récursivement les sous-termes
	à la bonne précision (selon l'opération à réaliser). La preuve se fait automatiquement, il faut juste préciser le variant.
	\begin{minted}{ocaml}
let rec compute (t:term) (n:int) : int
	requires { wf_term t }
	ensures { framing (interp t) result n }
	variant { t }
	=
	match t with
	| Cst c     -> compute_cst n c
	| Add t1 t2 -> compute_add n (interp t1) (compute t1 (n+1)) (interp t2) (compute t2 (n+1))
	| Neg t     -> compute_neg n (interp t)  (compute t n)
	| Sub t1 t2 -> compute_sub n (interp t1) (compute t1 (n+1)) (interp t2) (compute t2 (n+1))
	| Sqrt t    -> compute_sqrt n (interp t) (compute t (2*n))
	end
	\end{minted}

	\paragraph{22. two properties about euclidian division } J'ai hésité quant à la manière d'écrire ces propriétés.
	En effet, chacune se décompose en réalité en deux parties. Malgré celà, j'ai finalement décidé de ne faire que deux lemmes,
	chacun traitant des deux possibilités. Je les ai écrit comme des \mintinline{ocaml}{function lemma}, mais je n'ai finalement
	pas eu besoin de détailler la preuve: Z3 a su prouver ces deux lemmes sans indication.

	\paragraph{23. inv\_simple\_simple} C'est la fonction que j'ai eu le plus de mal à prouver. En plus des deux propriétés précédentes,
	j'ai ajouté quelques lemmes triviaux pour aider les solveurs pour la suite.
	Posons $d = B^{n+k} // p$ (avec $//$ la division euclidienne) et $r = B^{n+k} \% p$ (avec $\%$ le reste euclidien).
	La première étape, commune aux deux branches \mintinline{ocaml}{if} et \mintinline{ocaml}{else},
	consiste à montrer que $B^{n+k} // (p+1) < \frac{1}{x} \times B^n < (B^{n+k} // (p-1)) + 1$.
	Pour cela, j'utilise l'encadrement initial de $x$ et je passe à l'inverse (le reste suit assez facilement).
	Je montre également les deux propriétés suivantes qui seront utiles par la suite: $B^{n+k} \leq \frac{p^2}{4}$
	et $d \leq \frac{p-1}{2}$. La première se montre facilement en remarquant que $B^k \leq p$, et la seconde découle
	de la première (on majore $dp$ puis on divise par $p$).\\

	Le cas du \mintinline{ocaml}{if} est le plus compliqué. Il faut montrer que $B^{n+k} // (p-1) = d$,
	puis conclure en combinant cela avec l'inégalité prouvée précedemment: $B^{n+k} // (p+1) < \frac{1}{x} \times B^n < (B^{n+k} // (p-1)) + 1$.
	Pour prouver $B^{n+k} // (p-1) = d$, on peut montrer que $dp + rp + p \leq p^2$ en se servant de l'hypothèse du if
	($ 2r \leq p $) et de $B^{n+k} \leq \frac{p^2}{4}$, puis on divise le tout par $p$ (ce qui donne $d + r < p$). On peut alors
	utiliser l'une des deux propriétés montrées à la question 22 pour obtenir le résultat souhaité.\\

	Pour le \mintinline{ocaml}{else}, on peut facilement montrer que $B^{n+k} // (p+1) = d$ en utilisant le fait que $d \leq \frac{p-1}{2}$
	pour montrer que $r >= d$, puis en utilisant l'autre propriété montrée à la question 22.
	On conclut encore une fois en combinant ce résultat avec $B^{n+k} // (p+1) < \frac{1}{x} \times B^n < (B^{n+k} // (p-1)) + 1$.

	\paragraph{24. inv\_simple} La preuve de cette fonction repose sur les garanties de \mintinline{ocaml}{inv_simple_simple}.
	La seule chose à faire est de prouver la précondition de cette denière (il suffit pour cela de multiplier l'encadrement
	initial de $x$ par $B^m$ et de simplifier), puis de déduire de ses garanties l'encadrement voulu (il faut encore une fois
	multiplier l'encadrement par $B^m$ pour cela).

	\paragraph{25. term', interp', wf\_term'} Aucune difficulté pour cette question, \mintinline{ocaml}{wf_term} se complète ainsi:
	\begin{minted}{ocaml}
Inv' t     -> (wf_term' t) && (interp' t <>. 0.)
	\end{minted}

	\paragraph{26. correction} La correction de \mintinline{ocaml}{compute'} et \mintinline{ocaml}{msd} est presque automatique.
	Seules quelques étapes intermédiaires triviales sont requises pour le cas du \mintinline{ocaml}{else} de \mintinline{ocaml}{Inv'}
	et celui de \mintinline{ocaml}{msd}.

	\paragraph{27. termination} Pour prouver la terminaison de ces deux fonctions, j'ai commencé par créer une fonction qui calcule
	le variant de \mintinline{ocaml}{msd} suggéré par l'énoncé:
	\begin{minted}{ocaml}
let ghost function msd_variant (t:term') (n:int) : int =
	let a = - (floor (log2 (abs (interp' t)))) + 1 in
	if a < n then 0 else a - n
	\end{minted}
	Pour prouver la terminaison des deux fonctions (qui sont mutuellement recursives),
	j'ai donc combiné ce variant avec celui de la première version de \mintinline{ocaml}{compute}:
	\mintinline{ocaml}{variant { t, msd_variant t n }}.
	Why3 interprète cela avec un ordre lexicographique, et c'est bien ce que nous voulons ici.\\

	Pour prouver la stricte décroissance de ce variant, le seul cas qui pose problème est le cas du \mintinline{ocaml}{if}
	de \mintinline{ocaml}{msd}. Il faut prouver que, si cette branche est atteinte, alors \mintinline{ocaml}{msd_variant t n}
	est strictement positif. La première chose à faire est de majorer \mintinline{ocaml}{abs (interp' t)} par $2B^{-n}$,
	ce qui se fait assez facilement en utilisant l'encadrement de l'interpretation de $t$ et le fait que $c=0 \vee c=1 \vee c=2$
	(il y a plusieurs cas à traiter, mais Why3 s'en sort très bien). Il ne reste alors plus qu'à passer cette inégalité au log
	pour faire apparaitre la valeur du $a$ de \mintinline{ocaml}{msd_variant t n} et obtenir le résultat souhaité.
	L'axiome ajouté au tout début sert pour le passage au log de cette question.

	\section{Conclusion}

	J'ai trouvé le sujet un peu long et redondant. Notamment, beaucoup de preuves se ressemblaient et consistaient principalement
	à faire des simplifications élémentaires sur des inégalités (qui souvent ne posent aucun problème sur le papier,
	mais peuvent se révéler assez longues pour les faire accepter par Why3, ce qui n'est pas toujours très intéressant/motivant).
	Le fait de devoir recopier les preuves une seconde fois dans le rapport accentue ce coté redondant (bien que pour ma part
	j'ai préféré donner les idées plutot que de réelles preuves, afin de respecter le nombre de pages indiqué).
	Malgré cela, j'ai trouvé que d'autres preuves étaient assez intéressantes
	(notamment, la preuve de terminaison de la version finale de \mintinline{ocaml}{compute}).
	Finalement, je trouve que ce sujet permet tout de même de prouver un résultat intéressant,
	et m'a permis de bien progresser avec Why3.\\
	
	Concernant Why3, j'ai eu beaucoup de mal au début à me familiariser avec le système de preuve.
	Je n'arrivais pas vraiment à prévoir ce que les solveurs allaient pouvoir prouver ou non,
	et j'avais du mal à trouver quoi faire pour les aider. Avec le temps j'ai mieux compris comment faire des preuves plus compréhensibles
	par les solveurs, et j'arrivais donc plus rapidement à passer de la preuve papier à une preuve Why3 fonctionnelle.
	Notamment, j'ai vite compris l'importance de ne pas faire trop de résultats intermédiaires pour ne pas polluer l'environnement
	(préferer \mintinline{ocaml}{by ... so ...} à une succession d'assertions).
	Cependant, j'ai trouvé les capacités des solveurs très inégales, certains résultats non triviaux
	se prouvant parfois automatiquement (la question 22 avec Z3 par exemple) alors que des opérations élémentaires
	sur les inégalités pouvant mettre à mal tous les solveurs (surtout lorsqu'il s'agit d'inégalités strictes).\\

	Voici quelques suggestions:
	\begin{enumerate}
		\item Certains solveurs tels que Why3 se "noient" très vite dans les hypothèses et résultats intermédiaires:
		ils peuvent se réveler très efficaces pour prouver un résultat toplevel tel qu'un lemme,
		mais deviennent incapables de prouver ce même résultat lorsque placé au milieu d'une fonction.
		Cela est compréhensible, néanmoins j'aurais parfois aimé avoir plus de controle là-dessus.
		Il y a bien la commande \mintinline{ocaml}{clear_but} qui donne du controle, mais je ne la trouve pas très pratique à utiliser
		(d'ailleurs je ne l'ai pas utilisé) et j'aurais préféré qu'il y ai dans le langage un mot clé tel que \mintinline{ocaml}{using}
		qui permette de spécifier des lemmes/axiomes à utiliser en priorité, sans pour autant vider intégralement le reste de l'environnement.
		\item L'utilisation explicite de la fonction \mintinline{ocaml}{from_int} rend les preuves assez lourdes et elles perdent beaucoup en lisibilité.
		Pour un projet tel que celui-là, il aurait été agréable qu'il y ai une option pour inférer automatiquement les endroits où une conversion
		entier vers réel est nécessaire.
		\item L'éditeur a parfois quelques lacunes, notamment l'absence de CTRL+F et le fait que recharger l'environnement efface l'historique
		de CTRL+Z. D'ailleurs, il serait très agréable de pouvoir utiliser le moteur de Why3 dans un éditeur de texte plus conventionnel, tel
		que Visual Studio Code (comme il existe deja une version web de Why3, faire une extension pour Visual Studio Code ou même Atom 
		ne devrait pas être très long).
	\end{enumerate}

\end{document}