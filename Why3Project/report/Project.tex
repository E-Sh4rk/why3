% Header
\documentclass[a4paper]{article}%      autres choix : book, report
\usepackage[utf8]{inputenc}%           gestion des accents (source)
\usepackage[T1]{fontenc}%              gestion des accents (PDF)
\usepackage[francais]{babel}%          gestion du français
\usepackage{textcomp}%                 caractères additionnels
\usepackage{mathtools,amssymb,amsthm}% packages de l'AMS + mathtools
\usepackage{lmodern}%                  police de caractère
\usepackage[top=2cm,left=2cm,right=2cm,bottom=2cm]{geometry}%     gestion des marges
\usepackage{graphicx}%                 gestion des images
\usepackage{array}%                    gestion améliorée des tableaux
\usepackage{calc}%                     syntaxe naturelle pour les calculs
\usepackage{titlesec}%                 pour les sections
\usepackage{titletoc}%                 pour la table des matières
\usepackage{fancyhdr}%                 pour les en-têtes
\usepackage{titling}%                  pour le titre
\usepackage{enumitem}%                 pour les listes numérotées
\usepackage{hyperref}%                 gestion des hyperliens
\usepackage{minted}
\usemintedstyle{vs}

\hypersetup{pdfstartview=XYZ}%         zoom par défaut

\setlength{\droptitle}{-5em}   % This is your set screw
\title{\vspace{1.5cm}Projet Why3}
\author{Mickaël LAURENT}
\date{\vspace{-5ex}}

\pagenumbering{gobble}

\begin{document}

	\maketitle

	\section{Implémentation et preuves}

	\paragraph{0. Axiomes}
	Je n'ai eu besoin que de l'axiome proposé dans le sujet:
	\[\forall xy, 0 < x < y \implies log(x) < log(y)\]

	\paragraph{1. power2}
	J'avais initialement implémenté cette fonction de manière itérative (code toujours présent mais commenté),
	mais j'ai finalement opté pour une version récursive, à la fois plus concise à implémenter et à prouver.
	En effet, la bibliothèque standard définie elle-même power de manière récursive, rendant la preuve de mon implémentation immédiate.
	Il a seulement été nécessaire d'indiquer le variant: l'argument $l$ en entrée de la fonction.

	\paragraph{2. shift\_left}
	J'ai opté pour l'implémentation suivante:  \mintinline{ocaml}{z * (power2 l)}.\\
	Elle traduit littéralement la spécification de la fonction: \mintinline{ocaml}{result = z * (power 2 l)}.

	\paragraph{3. ediv\_mod}
	Mon implémentation est itérative et utilise la méthode d'euclide, comme celle faite en exercice durant l'année.
	En revanche, ici $x$ peut être négatif, ainsi j'utilise deux implémentations différentes selon le cas:
	si $x \geq 0$ alors je commence avec un reste $r$ positif auquel je soustrait $y$ à chaque itération, sinon, si $x < 0$,
	je commence avec un $r$ négatif auquel j'ajoute $y$ à chaque itération.\\
	La terminaison se prouve facilement en spécifiant le variant: $r$ dans un cas, $-r$ dans l'autre.\\
	La correction se prouve également sans difficulté en utilisant l'invariant adéquate, par exemple pour le cas $x < 0$:
	\mintinline{ocaml}{x = !q * y + !r /\ !r < y}.

	\paragraph{4. shift\_right}
	J'utilise l'implémentation suivante: \mintinline{ocaml}{let d,_ = ediv_mod z (power2 l) in d},
	qui traduit directement la spécification voulue: \mintinline{ocaml}{result = ED.div z (power 2 l)}.
	
	\paragraph{5. isqrt}
	Là encore j'utilise l'implémentation itérative vue en cours:
	\begin{minted}{ocaml}
let count = ref 0 in
let sum = ref 1 in
while !sum <= n do
	invariant { 0 <= !count /\ !count * !count <= n /\ !sum = (!count+1)*(!count+1) }
	variant { n - !sum }
	count := !count + 1;
	sum := !sum + 2 * !count + 1
done ; !count
	\end{minted}
	La terminaison est immédiate étant donné le variant,
	mais la correction (\mintinline{ocaml}{result = floor (sqrt (from_int n))}) nécessite tout de même
	de spécifier quelques trivialités après la boucle while (voir dans le code).\\

	NOTE: je n'ai pas eu besoin du lemme suggéré (euclid\_uniq) pour les 5 questions précédentes.

	\paragraph{6-12. some properties} La quasi-totalité des propriétés de l'énoncé ont été prouvées
	immédiatement par Z3, excepté \mintinline{ocaml}{_B_9} et \mintinline{ocaml}{_B_11} qui ont
	nécessité une étape intermédiaire (lemmes \mintinline{ocaml}{_B_9_aux} et \mintinline{ocaml}{_B_11_aux}).
	\begin{minted}{ocaml}
lemma _B_6 : forall n. _B n >. 0.
lemma _B_7 : forall n, m. (_B n) *. (_B m) = _B (n + m)
lemma _B_8 : forall n. _B n *. _B (-n) = 1.
lemma _B_9_aux : forall n. sqrt (_B (2*n)) = _B n
lemma _B_9 : forall a,n. 0. <=. a -> sqrt (a *. _B (2*n)) = (sqrt a) *. (_B n)
lemma _B_10 : forall y. 0 <= y -> _B y = from_int (power 4 y)
lemma _B_11_aux : forall y. (pow b y)*.(pow b (-.y)) = 1. (* Same as B_8 but for reals *)
lemma _B_11 : forall y. y < 0 -> _B y = inv (from_int (power 4 (-y)))
lemma _B_12 : forall y. 0 <= y -> power 2 (2 * y) = power 4 y
	\end{minted}

	\paragraph{13. framing} La définition de l'énoncé est meilleure que $|x - p(B^{-n})| < B^{-n}$ car elle
	est divisée en deux inégalités plus simples que l'on peut prouver séparemment.
	Ces deux inégalités auraient de toute façon dû être prouvées avant de pouvoir prouver $|x - p(B^{-n})| < B^{-n}$
	car une disjonction de cas est requise par la valeur absolue.

	\section{Conclusion}
		
	\begin{enumerate}
		\item Lemmes en dehors des fonctions: plus efficace pour prouver des résultats mathématiques indépendants.
	\end{enumerate}

\end{document}